\chapter{Conclusão}

\section{Considerações Finais}
Neste artigo, exploramos a criação de uma plataforma online inovadora e projetada para atuar como uma importante ferramenta no estímulo a doação de leite humano. 
Ao longo do processo de desenvolvimento, reconhecemos que o sucesso da plataforma depende da conscientização e da participação ativa das usuárias. Portanto, além de fornecer informações úteis e relevantes acerca do fluxo e importância da doação, é fundamental criarmos estratégias de divulgação eficazes para alcançar um público mais amplo e incentivar mais mulheres a se tornarem doadoras.

Este foi apenas o primeiro passo em direção a uma aplicação mais abrangente e impactante. Será fundamental a realização de avaliações contínuas, obtenção de \textit{feedbacks} das usuárias e implementação de melhorias constantes a fim de aprimorar a experiência de usabilidade da plataforma e assim alcançar resultados ainda mais significativos no campo da doação de leite humano.

Por fim, acreditamos firmemente que o produto proposto tem um potencial significativo para promover mudanças reais e concretas. Ao incentivar um maior número de mulheres a se tornarem doadoras, a aplicação tem o poder de contribuir de forma efetiva na redução do índice de mortalidade infantil e na melhoria da saúde e bem-estar dos recém-nascidos que dependem, em muitos casos, da generosidade de terceiros para acessar esse valioso recurso alimentar. 

\section{Trabalhos Futuros}
Nos concentramos em garantir um mínimo produto viável neste primeiro momento. No entanto, nosso produto já conta com algumas ideias de novas funcionalidades que serão desenvolvidas em um momento posterior. A seguir listaremos algumas delas.

\subsubsection{Participação Ativa dos Bancos de Leite}
A princípio, a funcionalidade de registrar uma nova doação, funciona para a usuária apenas como uma ferramenta de controle e visualização. Posteriormente, a aplicação objetiva contar com a participação ativa dos Bancos de Leite e Postos de Coleta do Recife. Incluindo-os em nosso sistema a fim de que os mesmos sejam responsáveis por confirmar que a doação de leite registrada pela usuária de fato aconteceu. Trazendo dessa forma mais credibilidade aos dados.
Além disso, buscaremos implementar uma integração direta ao sistema dos BLHs da cidade do Recife que atualize semanalmente (ou em menor tempo conforme necessidade da unidade) a quantidade de seus estoques para que esta possa ser exibida na plataforma. Hoje, a inserção desse dado será feita de forma mecânica.

\subsubsection{Rede de doação e troca de itens infantis}
Sabemos que o momento do aleitamento e doação é finito. Portanto, como forma de manter as usuárias do site e trazer novas pessoas para perto, estamos desenvolvendo uma funcionalidade que permitirá que qualquer pessoa usuária possa inserir em nossa aplicação, itens infantis em bom estado que desejam doar. Bem como, poderão sinalizar a pessoa que está doando aquele item, que possuem interesse nele. Promovendo assim, um consumo mais consciente de produtos que muitas vezes se perdem muito rapidamente, dado o acelerado crescimento dos bebês.

\subsubsection{Recomendações de profissionais de saúde}
Na área da doadora, como mais uma forma de monetizar nossa plataforma, ambicionamos ter uma nova seção onde profissionais de saúde relacionados a esse nicho, por exemplo: consultoras de amamentação, pediatras etc, através do pagamento de um valor a ser definido, possam ter seus trabalhos exibidos e indicados.