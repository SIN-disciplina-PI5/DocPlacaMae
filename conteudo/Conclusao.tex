\chapter{Conclusão}

\section{Considerações Finais}
Com este trabalho buscamos explorar como o cyberbullying tem crescimento significativo entre os adolescentes e jovens e como a criação desta plataforma educativa é de extrema importância para atuar como uma ferramenta de combate a este tipo de violência. Juntamente com um material educativo dentro da plataforma online, procuramos mostrar a devida relevância de reconhecer esse tipo de violência e de como agir a partir deste tipo de acontecimeno, quais os direitos e atitudes a serem tomadas a partir disto.\\

Ao longo do processo de desenvolvimento, reconhecemos que o sucesso da plataforma depende da conscientização e da participação ativa dos usuários. Portanto, além de fornecer informações úteis e relevantes acerca dos procedimentos de como agir a partir de uma situação como esta, também deixamos a disposição um quiz de conscientização para enfatizar todas as informações que estão disponíveis na plataforma.\\


Este foi apenas o ponto de partida para dar ênfase a conscientização ao combate a este tipo de violência. É de suma importância acompanhar as interações dos usuários com a plataforma, obter feedbacks e assim conseguir alcançar resultados positivos em relação a diminuição de desinformação e até de casos de cyberbullying.\\


Por fim, acreditamos firmemente que o produto proposto tem um potencial significativo para promover mudanças reais e concretas. Ao buscar informar um número significativo de adolescentes e jovens, conscientizando-os sobre o cyberbullying, esta aplicação consegue contribuir efetivamente na redução de casos, no aumento dos índices de denuncias e na procura de ajuda, contribuindo com o aumento de informação sobre esta temática.
