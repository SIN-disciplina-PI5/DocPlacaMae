\chapter{Introdução}


\section{Inicio da Contextualização e Motivação}

Segundo estudos realizados em escolas de São Paulo em junho de 2020, a crescente integração da tecnologia em ambientes educacionais tem sido acompanhada por um aumento preocupante na incidência de cyberbullying. Este fenômeno, caracterizado pela prática de agressões, humilhações e ameaças por meio de dispositivos eletrônicos e redes sociais, tem impactos significativos na saúde mental e no bem-estar dos estudantes. Este documento busca fornecer uma visão abrangente sobre o cyberbullying em escolas brasileiras, com base em pesquisas realizadas em diversas regiões do país.

De acordo com os dados coletados no ultimo ano, aproximadamente 37\% dos alunos investigados estavam envolvidos em situações de cyberbullying. Esses casos variaram desde vitimização exclusiva até participação como agressores ou vítimas-agressores. Além disso, os resultados revelaram diferenças significativas de gênero nas formas de cyberbullying experimentadas, destacando a importância de estratégias de intervenção sensíveis ao contexto.

Nesse contexto, a plataforma de Quiz desenvolvida pelo site PlacaMãe.Org surge como uma ferramenta promissora para enfrentar o cyberbullying e promover uma cultura de respeito e inclusão nas escolas.
Ao gamificar o processo de aprendizagem, a plataforma de Quiz oferece uma maneira envolvente e interativa de educar os alunos sobre a importância do respeito online e do comportamento ético nas redes sociais. Além disso, ao fornecer informações e recursos educacionais sobre o cyberbullying de forma acessível e atraente, a plataforma capacita os alunos a reconhecerem, prevenirem e denunciarem situações de violência virtual.

 

\section{Problemática}
O Bullying se encontra em relevância no cenário atual e a sua abordagem é crucial,visto que está marcado por transformações sociais e tecnológicas profundas, especialmente com o uso da internet por crianças, adolescentes e jovens, conhecidos como "nativos digitais". Essa imersão tecnológica requer uma reavaliação das estruturas sociais, desde a relação com o conhecimento até as interações interpessoais. Nesse contexto, surge como ponto central o fenômeno do cyberbullying, uma forma de violência presente no ambiente virtual com impactos significativos no bem-estar e saúde mental dos jovens.

A preocupação com o uso inadequado da internet e suas consequências negativas, como o cyberbullying, desafia o ordenamento jurídico a regulamentar essas questões de maneira eficaz. Destaca-se a importância de compreender o papel da escola, como parte da rede de proteção infantojuvenil, na prevenção e combate a essa forma de violência. A escola é considerada um espaço privilegiado para a implementação de ações educativas e preventivas, visando conscientizar os jovens sobre os riscos associados ao uso inadequado da internet e formas de prevenir o cyberbullying.

Além disso, ressalta-se a necessidade de implementação de políticas públicas e programas específicos, conforme estabelecido na Lei 13.185/2015, que criou o Programa de Combate à Intimidação Sistemática. Essa legislação representa um marco na luta contra o cyberbullying, propondo medidas preventivas e restaurativas para lidar com os conflitos. No entanto, sua eficácia depende da articulação efetiva da rede de proteção à infância e adolescência, envolvendo não apenas as escolas, mas também famílias, comunidades e órgãos governamentais.

Assim, o evento destaca a complexidade do fenômeno do cyberbullying e a necessidade urgente de uma abordagem multidisciplinar e integrada para combatê-lo. É crucial unir esforços para estabelecer um conjunto de normas que protejam efetivamente a infância e a adolescência brasileira, promovendo um ambiente seguro e inclusivo, tanto no mundo físico quanto virtual.

Em resumo, as principais questões a serem resolvidas consistem em: 
\begin{itemize}
    \item O Controle do acesso a informação disponível para os menores de idade
    \item A forma de acompanhamento dos pais com as crianças fora da escola
    \item A criação de uma legislação mais assertiva e especifica para lidar com esse tipo de caso
\end{itemize} 

\section{Objetivos}


\subsection{Objetivo Geral}

Incentivar o aumento de doações de leite materno nos bancos de leite do estado de Pernambuco.


\subsection{Objetivos Específicos}

\begin{itemize}
  \item Conscientizar nutrizes acerca da importância e dos benefícios da doação do leito materno.
  \item Disponibilizar o quantitativo dos estoques dos BLH atualizados semanalmente, para que as doadoras tenham ciência das unidades com maior necessidade e assim possam direcionar melhor sua doação.
  \item "Gameficar" a aplicação a fim de promover um maior engajamento por parte dos usuários.
\end{itemize}


 