\chapter{}

\section{Referências}

Cetic.br (2024). TIC Kids Online Brasil. Visitado em 10 de abril de 2024, de\\ https://cetic.br/pesquisa/kids-online \\

Parentes da Costa, G. O., Silva, F. M., Oliveira, R. A., & Almeida, T. C. (2023). Ciberbullying: a inovação maléfica nas práticas de violência escolar na adolescência. Brazilian Journal of Surgery & Clinical Research, 44(3), p. 132. \\

Passarelli, B., Junqueira, H. J., & Angelucci, A. C. B. (2014). Os nativos digitais no Brasil e seus comportamentos diante das telas. Matrizes, 8(1), 159-178. Universidade de São Paulo, São Paulo, Brasil. \\

Ribeiro, B. G., Souza, N. H. C., Andrade, N. S., & Santos, P. L. (2021). Saúde e doença no processo de envelhecimento. Universidade Nove de Julho. \\https://bibliotecatede.uninove.br/handle/tede/2959 \\

SaferNet.org.br (s.d.). SaferNet Brasil. Visitado em 10 de abril de 2024, de \\https://new.safernet.org.br/. \\

Zafani, G. S. (2021). Políticas públicas federais e estaduais para prevenção e contenção ao bullying e cyberbullying no Brasil após a promulgação da Lei Federal 13.185/2015. 124 f. Dissertação de Mestrado, Universidade Estadual Paulista.

