\chapter{Trabalhos Relacionados}

Algumas iniciativas similares foram identificadas no campo de estudo deste trabalho, as quais serão apresentadas a seguir. Estas poderão ser utilizadas como referências importantes para aprimoramento e desenvolvimento da nossa aplicação.\\
\begin{enumerate}
    \item O aplicativo de celular \emph{Amamenta Brasília}, disponível para Android e IOS, funciona desde 2017. Criado pela Secretaria de Saúde do Distrito Federal, o app tem o objetivo de incentivar e apoiar a amamentação. Funciona fornecendo informações sobre aleitamento materno, materiais educativos e informações sobre bancos de leite. Uma funcionalidade importante do site é a possibilidade de agendamento para a coleta em domicílio do leite humano ordenhado pelas mulheres que desejam doá-lo. A busca do alimento é realizada pelo corpo de bombeiros do DF que o encaminha aos bancos de leite da região. A página do programa \cite{amamentabrasilia} traz mais informações
    \item Em 2018 foi lançado no Ceará, por  \cite{Carvalho}, o aplicativo para celular \emph{Amamente e Doe}. A aplicação propõe uma solução semelhante ao Amamenta Brasília, porém, sem envolvimento do estado. O aplicativo reunia informações sobre a importância da doação de leite, benefícios do aleitamento materno e auxiliava no processo da coleta de LH e armazenamento do mesmo. Bem como orientava como proceder para efetivar a doação e identificava a localização dos postos de coleta e bancos de leite humano (BLHs) do estado. Infelizmente, por motivos desconhecidos, o aplicativo não encontra-se mais em funcionamento.
    \item No ano de 2022, o então deputado Francisco Jr (PSD-GO), apresentou à câmara o Projeto de Lei(PL) 870/22 o qual instituía a criação de um Banco Virtual de Leite Materno. A ideia era desenvolver um aplicativo para dispositivos móveis para permitir que as doadoras de leite humano tivessem acesso ao sistema de gerenciamento dos bancos de leite da rede pública de seu estado As usuárias poderiam agendar através do app a retirada do leite já ordenhado pelo agente público responsável pela coleta domiciliar, além de ter acesso a informações sobre os procedimentos para adequada coleta e conservação do leite. O PL \cite{PL870/2022} atualmente aguarda tramitação no senado.
\end{enumerate}

Nossa proposta de aplicação, por sua vez, se concentrará em além de trazer toda gama de informações já citadas nas plataformas acima, desenvolver uma solução tecnológica"\textit{gameficada}", a fim de deixar a experiência da doação de leite humano mais interessante.




