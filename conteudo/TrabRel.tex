\chapter{}

\subsection{Trabalhos Relacionados}

Alguns sites similares foram identificados no campo de estudo deste trabalho, as quais serão apresentadas a seguir. Estes poderão ser utilizados como referências importantes para aprimoramento e desenvolvimento do nosso site.

\begin{enumerate}
    \item A SaferNet Brasil é uma associação civil de direito privado, com atuação nacional, sem fins lucrativos ou econômicos, sem vinculação político partidária, religiosa ou racial. Fundada em 20 de dezembro de 2005, com foco na promoção e defesa dos Direitos Humanos na Internet no Brasil, voltada à prevenção e ao combate a crimes contra os direitos humanos na internet, defesa, responsabilização, prevenção, capacitação e formação, pesquisa e desenvolvimento e mobilização social. A página do \cite{safernet} traz mais informações.
    
    \item \emph{O Centro Regional de Estudos para o Desenvolvimento da Sociedade da Informação}, departamento do Núcleo de Informação e Coordenação do Ponto BR (NIC.br), ligado ao Comitê Gestor da Internet do Brasil (CGI.br), tem a missão de monitorar o acesso, o uso e a apropriação das tecnologias de informação e comunicação (TIC) no Brasil desde 2005, objetivo cumprido por meio da produção de indicadores sobre o acesso, o uso e a apropriação das TIC em vários segmentos da sociedade. Tais dados servem como insumo para o desenho e o monitoramento de políticas públicas que contribuam para o desenvolvimento da Internet no país. E um dos seus nichos é a área de TIC Kids Online Brasil que tem como objetivo gerar evidências sobre o uso da Internet por crianças e adolescentes no Brasil. Realizada desde 2012, a pesquisa produz indicadores sobre oportunidades e riscos relacionados à participação on-line da população de 9 a 17 anos no país.
    
\end{enumerate}

Nossa proposta de aplicação, por sua vez, se concentrará em além de trazer toda gama de informações já citadas nas plataformas acima, desenvolver uma solução tecnológica \textit{gamificada}, a fim de deixar os estudos mais interativos.
